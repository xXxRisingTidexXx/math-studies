\documentclass[mathserif]{beamer}

\usepackage{lmodern}
\usepackage{graphicx}
\usepackage{wrapfig}
\usepackage[T2A]{fontenc}
\usepackage[utf8]{inputenc}
\usepackage[ukrainian]{babel}
\usepackage{afterpage}
\usepackage{placeins}
\usepackage{cmap}
\usepackage[labelsep=period]{caption}
\usepackage{subcaption}
\usepackage{hyperref}
\usepackage{mathtools}

\setbeamertemplate{navigation symbols}{}
\usetheme{Darmstadt}
\usecolortheme{crane}

\hypersetup{colorlinks=true,linkcolor=blue,urlcolor=blue,citecolor=blue,anchorcolor=blue}

\title{Стохастичний градієнтний спуск}
\institute{Державна наукова установа \\ ``Київський академічний університет``}
\author[Д.О.~Петраківський]{\textbf{Данило Олександрович Петраківський} \\ {\textsl{d.petrakivskyi@kau.edu.ua}}}
\date{\today}
\logo{\includegraphics[height=10mm]{images/kau}\hspace{7pt}}

\begin{document}
    \frame{\titlepage}

    \begin{frame}{Стохастичний градієнтний спуск}
        \href{https://en.wikipedia.org/wiki/Stochastic_gradient_descent}{Стохастичний градієнтний спуск} ---
        ітеративний метод оптимізації градієнтного спуску за допомогою стохастичного наближення.
        Використовується для прискорення пошуку цільової функції шляхом використання обмеженого за розміром
        тренувального набору, який вибирається випадково при кожній ітерації.
    \end{frame}

    \begin{frame}{Приклад}
        Знайдемо алгоритм $a(x, w)$, що апроксимує залежність $y ^ *$.
        У випадку лінійного класифікатора шуканий алгоритм має вигляд:

        \[a(x, w) = \varphi \left(\sum_{j=1} ^ n w_j x ^ j - w_0 \right)\]
    \end{frame}

    \begin{frame}{Тіло методу}
        \begin{enumerate}
            \item Ініціалізувати ваги $w_j$, ($j = 0, \dots, k$), де $k$ --- розмірність простору ознак.
            \item Ініціалізувати поточну оцінку функціонала:
                \[Q \coloneqq \sum_{i=1} ^ n L(a(x_i, w), y_i)\]
            \item Поки значення $Q$ не стабілізується та/або ваги $w$ не припинять змінюватись.
        \end{enumerate}
    \end{frame}

    \begin{frame}
        \begin{center}
            \Huge \bfseries \itshape Дякую за увагу!
        \end{center}
    \end{frame}
\end{document}
